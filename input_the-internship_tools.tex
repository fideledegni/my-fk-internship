Pendant toute la durée du stage, j'ai travaillé sur une machine Linux sur laquelle j'avais en local la base de code de l'entrprise. La base de code de Fund KIS est découpée en plusieurs \textit{repositories} Git Hub privés et il y avait un outil presqu'automatique pour récupérer les \textit{repositories} en local. Dès le premier jour de stage, nous (l'autre stagiaire et moi) avons été ajoutés à l'équipe de développement de l'entreprise sur Git Hub pour avoir accès à la base de code. A partir de ce moment, nous pouvions cloner tous les \textit{repositories} nécessaires à notre travail.


\vspace{3mm}
Il y avait au moins deux branches par \textit{repository} : la branche \textit{master} qui est en production et la branche \textit{develop} sur laquelle les nouvelles fonctionnalités sont implémentées. Lorsque ce qui est sur cette dernière branche est testé et fonctionne, on peut l'envoyer sur la première. On s'assure ainsi que le code en production est toujours fonctionnel. Mon travail était intégré aux autres développements de Fund KIS car j'avais besoin d'utiliser d'autres \textit{packages} déjà développés par l'entreprise. Même si j'ai écrit de nouveaux \textit{packages} (qui nécessitaient de données d'autres \textit{packages} existants), j'ai eu parfois à ajouter directement des choses dans des \textit{packages} déjà développés. Je travallais en local puis après avoir testé mon code, je pouvais le \textit{pusher} sur la branche \textit{develop} de la base de code sur Git Hub.

\vspace{3mm}

Le travail a été effectué en \textit{JavaScript} et beaucoup de \textit{frameworks} ont été utilisés dont NodeJS, AngularJS, Mongo, React. Je connaissais déjà un peu \textit{JavaScript}, NodeJS et AngularJS mais j'ai beaucoup appris sur ces technologies au fur et à mesure du déroulement de mon stage. Pendnat le stage, j'avais besoin de beaucoup lire aussi dans la base de code de Fund KIS car il fallait chercher parfois comment utiliser un bout de code et comprendre comment certains algorithmes déjà écrits fonctionnent. En effet, pour faire un travail qui s'intègre dans les développements de l'entreprise, il fallait éviter de refaire des choses qui ont déjà été faites pour ne pas avoir de codes dupliqués. Ces moments de lecture étaient aussi très précieux pour découvrir de nouvelles choses et apprendre comment bien structurer son code.
