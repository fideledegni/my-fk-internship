\documentclass[12pt]{article}
\usepackage[T1]{fontenc}
\usepackage[utf8]{inputenc}
\usepackage[french]{babel}
\usepackage{enumitem}
%\frenchbsetup{StandardLists=true} % Pour éviter de conflit avec enumitem qui traite aussi les itemize
%\usepackage{charter}
\usepackage{graphicx}
\usepackage{xcolor}
\usepackage{ulem, soul}
\usepackage{url}
\usepackage{eurosym}
\usepackage{tikz}
\usepackage{pifont} % font contenant des symboles sympas : appeler avec \ding{numéro}
\usepackage[top=4cm, bottom=3cm, left=3cm, right=2.5cm, headheight=62pt]{geometry}
\usepackage{fancyhdr}
\usepackage{hyperref}

\graphicspath{{./figs/}}
\makeatletter
%% meta function
\def\make@object#1{\expandafter\gdef\csname#1\endcsname{\object{#1}}}
\newcommand{\object}[1]{\texttt{#1}}
\make@object{forecast}
\make@object{bank}
\make@object{liquid}
\make@object{saving}
\make@object{operation}
\newcommand{\catForecast}[1]{\textsf{#1}}

%%% versions
\newcommand{\version}[3]{%
\setcounter{vmajor}{#1}
\setcounter{vmedium}{#2}
\setcounter{vminor}{#3}
}
\newcounter{vmajor}
\newcounter{vmedium}
\newcounter{vminor}
\newcommand{\theversion}{\thevmajor.\thevmedium.\thevminor}

\version{1}{0}{2}
\makeatother


\author{\textsc{Degni} Fidèle}
\date{Du 29 mai 2017 au 25 août 2017}
\title{Rapport de stage \\ Fund KIS}

\begin{document}

\maketitle
\thispagestyle{empty}

\newpage
\setcounter{page}{1} % Pour commencer la umérotation des pages ici
%\setcounter{tocdepth}{4} % Afficher jusqu'aux paragraphes dans le sommaire
\renewcommand{\contentsname}{Sommaire} % Pour remplacer <<Table des matière>> par <<Sommaire>>
\tableofcontents
\renewcommand{\thepage}{\arabic{page}}

\newpage
\section*{Remerciements}
\addcontentsline{toc}{section}{Remerciements}
Je tiens à remercier mon maître de stage, Hicham BOUHMADI, président de Fund KIS, de m'avoir permis de faire ce stage au sein de son entreprise, pour sa disponibilité et ses conseils.

\vspace{3mm}

Je remercie aussi Sylvain  PLESSIS, \textit{Research Software Engineer}, pour tout le temps qu'il m'a accordé lorsque j'avais des questions.

\vspace{3mm}

Mes remerciements aux responsables de la formation de l'EMSE ainsi qu'à mon tuteur école à l'EMSE, M. Olivier BOSSIER.

\vspace{3mm}

Enfin un clin d'\oe il Leticia, étudiante à l'EMSE et stagiaire-ingénieur avec qui j'ai travaillé sur le même sujet de stage.



\newpage
\section*{Introduction} % Pour ne pas numéroter l'intro comme une section
\addcontentsline{toc}{section}{Introduction} % Afficher l'intro dans la table des matières
Le stage d'application est une étape importante de la formation d'un étudiant en école d'ingénieurs, car il lui permet de mettre en application ses connaissances théoriques dans un contexte industriel. C'est aussi l'occasion d'acquérir une capacité à étudier, concevoir et mettre en \oe uvre des solutions scientifiques, techniques et technologiques ; et de développer les sept qualités de l'Ingénieur Civil des Mines (ICM) : interdisciplinarité, ouverture, discernement, audace, responsabilié, engagement et agilité.

\vspace{3mm}

J'ai effectué mon stage au sein de l'entreprise Fund KIS, éditeur de logiciels applicatifs pour la gestion d'actifs depuis 2011. Mon stage s'est déroulé dans le bureau de l'entreprise, où j'ai travaillé en tant qu'Ingénieur \'Etudes et Développement.

\vspace{3mm}

Je vais décrire dans les pages qui vont suivre le travail que j'ai réalisé pendant ce stage, son déroulement ainsi que ce que j'en ai appris.



\newpage
\section{Fund KIS - Présentation}
Fund KIS est une jeune entreprise innovante française créée en 2011 par un ancien de l'\'Ecole des Mines de Saint-\'Etienne, Hicham BOUHMADI (promotion 1994). Elle édite des logiciels applicatifs pour la gestion d'actifs et propose des solutions innovantes aux sociétés de gestion et leurs partenaires : calculs et suivi des indicateurs de risques, solutions de reporting financier, automatisation de documentations réglementaires, des solutions d'enrichissement des sites web avec des données fonds et des applications web ou mobile sur mesure. Elle a aussi développé le portails fonds \url{https://fundkis.com} qui propose de nombreux outils aux investisseurs. Son siège social est situé au 55 Bis avenue du Bois de Verrières - 92160 Antony.

\vspace{3mm}

La société est une société par actions simplifiée (SAS) dont l'effectif varie entre quatre et dix personnes et est dirigée depuis sa création par son président Hicham BOUHMADI. Elle a pour objectif de s'agrandir très bientôt pour répondre aux sollicitations clients de plus en plus grandes.



\vspace{1.5cm}
\section{Le stage au sein de l'entreprise}
\subsection{Objectifs du stage}
Fund Kis dispose d'un grand nombre de données qu'elle collecte ou calcule. La majorité de ces informations sont rendues publiques à travers son portail \url{https://fundkis.com}. La mission réalisée au cours du stage a consisté à rendre ces informations accessibles à travers une \textit{API REST} implémentant les meilleures pratiques, dont :
\begin{itemize}[font=\color{blue}, label=\ding{43}]
  \item API auto-documentée avec portail de documentation ;
  \item implémentation du protocol Swagger ou équivalent ;
  \item utilisation d'une API gateway gérant les problématiques de sécurité et de limites d'utilisations selon configurations ;
  \item architecture micro-service utilisant Docker ;
  \item architecture multi-API, c'est-à-dire dynamique et facilitant l'ajout de nouvelles API ;
  \item \textit{Continuous deployement}.
\end{itemize}

L'aboutissement du projet consistait à remplacer les API utilisées auparavant par le portail par ces nouvelles API. De plus, le projet devait s'intégrer aux autres développements de Fund KIS. Il devait notamment utiliser un ensemble de \textit{packages} communs : accès aux données, algorithmes et authentification. Les développement devaient être faits en \textit{fullstack Javascript}: Mongo, Node, React, Angular...

Comme le travail à effectuer était plutôt conséquent pour être réalisé sereinement par un seul stagiaire de deuxième année en trois mois, Fund KIS a recruté deux stagiaires. J'ai donc travaillé avec une autre étudiante de l'EMSE sur le sujet.


\subsection{Travail effectué}
Afin d'atteindre les objectifs définis par le cahier des charges du stage, le travail a été réalisé en plusieurs étapes.

\subsubsection{Documentation}
Fund KIS avait déjà repéré les technologies à utiliser pour réaliser la mission du stage. Il s'agit notamment de \href{https://getkong.org}{Kong} \footnote{voir \url{https://getkong.org}}, un \textit{API Gateway} open-source c'est-à-dire un gestionnaire d'API ; \href{https://swagger.io}{Swagger} \footnote{voir \url{https://swagger.io}}, un \textit{framework OAS (OpenAPI Specification)} pour gérer la documentation d'API et \href{https://www.docker.com}{Docker} \footnote{voir \url{https://www.docker.com}}, un gestionnaire de \textit{containers}. Au début du stage nous ne connaissions pas précisément à quoi servait chacun de ces différents outils. Nous nous sommes alors documentés sur chacunes de ces technologies ; c'était la première phase de la documentation. La suite a consité à étudier comment les utiliser pour réaliser notre travail.

\vspace{3mm}

Docker est déjà utilisé par Fund KIS pour le déploiement de ses applications dans des \textit{containers} en production. Un \textit{container} permet d'encapsuler une application en un format qui lui permet de tourner de façon indépendante. On peut le voir comme une machine virtuelle mais il n'est lié à aucun système d'exploitation. Il contient juste les bibliothèques et les paramètres permettant à l'application de tourner toute seule. Cela permet de s'assurer que l'application tournera quel que soit l'endroit où il est déployé puisque c'est un système complètement autonome qui contient tout ce dont il a besoin pour fonctionner.


\vspace{3mm}
Swagger est un \textit{framework} qui permet de suivre une certaine spécification pour gérer la documentation des API. Il permet d'expliquer simplement dans la documentation l'architechture \textit{RESTful} de l'API en détaillant les ressources et les opérations qu'on peut utiliser sur l'API. En utilisant \href{https://github.com/Surnet/swagger-jsdoc}{swagger-jsdoc}, on peut écrire toute la documentation de l'API comme commentaires dans le code source de l'API ; ceci est très intéressant car il permet d'avoir une documentation synchrone avec l'évolution de l'API. Les aspects concernant l'automatisation des documentations des API ont été traités par la deuxième statgiaire avec qui j'ai travaillé.

\vspace{3mm}

Kong est un \textit{API Gateway} et permet de gérer les problématiques de sécurité, d'authentification et de limites d'utilisations des API. On trouvera à la figure 1 suivante un schéma représentant le fonctionne de Kong et des \textit{API Gateway} en général. Il propose une interface \textit{RESTful} et utilise une architecture très modulaire en utilisant des \textit{plugins}. Ainsi on peut le configurer à souhait en utilisant seulement les \textit{plugins} dont on a besoin. On peut même écrire ses propres \textit{plugins} mais il faut pour cela maitriser Lua (langage de programmation).

\begin{figure}[!h]
\centering
\includegraphics[width=\linewidth]{apigateway}
\caption{Structure d'un \textit{API Gateway}}
\end{figure}

\vspace{3mm}

Cependant, Kong impose une grosse contrainte : il faut absolument une base de données \textit{Cassandra} pour l'utiliser. Mais Fund KIS utilise des bases de données Mongo et SQL. Utiliser Kong comme gestionnaire d'API reviendrait à copier des données des bases données déjà utilisées par l'entreprise dans la bases de données \textit{Cassandra} que Kong va utiliser. Cela est est fastidieux et risque de poser des problèmes d'inconsistance lorsque des données sont mises à jour dans les bases. Même si Kong semblait être une solution adapté à nos problématiques, nous avons dû nous en passer.


\subsubsection{\textit{Rate limiter}}
Le rate limiter

\vspace{3mm}

Middleware qui va avec


\subsubsection{API auto-documentée}
Les navs




% \vspace{5mm}
%
% \begin{figure}[!h]
% \centering
% \includegraphics[width=\linewidth]{documentation}
% \caption{Caption}
% \end{figure}


\subsection{Environnement de développement et Outils utilisés}
Pendant toute la durée du stage, j'ai travaillé sur une machine Linux sur laquelle j'avais en local la base de code de l'entreprise. La base de code de Fund KIS est découpée en plusieurs \textit{repositories} Git Hub privés et il y avait un outil presqu'automatique pour récupérer les \textit{repositories} en local. Dès le premier jour de stage, nous (Leticia et moi) avons été ajoutés à l'équipe de développement de l'entreprise sur Git Hub pour avoir accès à la base de code. \'A partir de ce moment, nous pouvions cloner tous les \textit{repositories} nécessaires à notre travail.


\vspace{3mm}
Il y avait au moins deux branches par \textit{repository} : la branche \textit{master} qui est en production et la branche \textit{develop} sur laquelle les nouvelles fonctionnalités sont implémentées. Lorsque ce qui est sur cette dernière branche est testé et fonctionne, on peut l'envoyer sur la première. On s'assure ainsi que le code en production est toujours fonctionnel. Mon travail était intégré aux autres développements de Fund KIS car j'avais besoin d'utiliser d'autres paquets déjà développés par l'entreprise. Même si j'ai écrit de nouveaux paquets (qui nécessitaient de données d'autres paquets existants), j'ai eu parfois à ajouter directement des choses dans des paquets déjà développés. Je travallais en local puis après avoir testé mon code, je pouvais le \textit{pusher} sur la branche \textit{develop} de la base de code sur Git Hub.

\vspace{3mm}

Le travail a été effectué en \textit{JavaScript} et beaucoup de \textit{frameworks} ont été utilisés dont NodeJS, AngularJS, Mongo, React. Je connaissais déjà un peu \textit{JavaScript}, NodeJS et AngularJS mais j'ai beaucoup appris sur ces technologies au fur et à mesure du déroulement de mon stage. Pendant le stage, j'avais besoin de beaucoup lire aussi dans la base de code de Fund KIS car il fallait chercher parfois comment utiliser un bout de code et comprendre comment certains algorithmes déjà écrits fonctionnent. En effet, pour faire un travail qui s'intègre dans les développements de l'entreprise, il fallait éviter de refaire des choses qui ont déjà été faites pour ne pas avoir de codes dupliqués. Ces moments de lecture étaient aussi très précieux pour découvrir de nouvelles choses et apprendre comment bien structurer son code.




\vspace{1.5cm}
\section{Environnement de travail}
L'environnement de travail était très convivial pendant mon stage. Le premier jour du stage, notre tuteur nous a présenté brièvement les activités de l'entreprise pour situer le contexte du travail que nous avions à faire. Fund KIS était une petite entreprise de quatre personnes lorsque j'y ai fait mon stage. Nous étions deux stagiaires, ce qui qui faisait en tout six personnes. Il y avait deux espaces de travail : nous les deux stagiaires avions notre bureau avec un autre employé dans un espace et mon tuteur (Président de l'entreprise) et les deux autres employés avaient leur bureau dans l'autre espace. Je voyais donc tout le monde tous les jours et l'ambiace de travail était bien détendu.

\vspace{3mm}

Je pouvais poser facilement des questions à mon tuteur car il travaillait dans le bureau à côté et était très disponible pour mes interrogations. De plus je pouvais demander de l'aide Sylvain qui travaillais dans le même espace que nous les deux stagiaire. Au début c'était un peu difficile de se mettre à travailler et faire les tests directement dans la base de code déjà existante et j'avais tendance à tester ce que je faisais de façon complètement isolée. Mais après quelques entretiens avec mon tuteur, j'ai pu commencer à travailler en intégrant directement ce que je faisais dans les développements de l'entreprise.

\vspace{3mm}

Nous avons quelques fois, tous les six, déjeuné au restaurant à midi.



\newpage
\section*{Conclusion}
\addcontentsline{toc}{section}{Conclusion}
Le stage s'est très bien passé et j'ai appris pas mal de choses



% Bibliographie
\newpage
\subsection*{Bibliographie}
\addcontentsline{toc}{subsection}{Bibliographie}

\footnotesize
\begin{itemize}
\item \url{https://fundkis.com}
\item \url{https://www.linkedin.com/company-beta/2381547}
\item \url{https://www.docker.com}
\item \url{https://getkong.org}
\item \url{https://swagger.io}
\item \url{http://microservices.io/patterns/microservices.html}
\item \url{https://dzone.com/articles/microservices-architecture-what-when-how}
\item \url{https://fr.wikipedia.org/wiki/Representational_state_transfer}
\item \url{https://blog.nicolashachet.com/niveaux/confirme/larchitecture-rest-expliquee-en-5-regles}
\end{itemize}
\label{fin}

%\newpage
%\listoffigures % Table des figures
%\listoftables

% \newpage
% \pagenumbering{alph}
% \pagestyle{plain}
% \appendix
% \section{Schémas}
% Schémas
% \section{Tableaux}
% Tableaux

\end{document}
