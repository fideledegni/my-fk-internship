Fund Kis dispose d'un grand nombre de données qu'elle collecte ou calcule. La majorité de ces informations sont rendues publiques à travers son portail \url{https://fundkis.com}. La mission réalisée au cours du stage consistait donc à rendre ces informations accessibles à travers une \textit{API REST} implémentant les meilleures pratiques, dont :
\begin{itemize}
  \item API auto-documenté et portail de documentation
  \item Implémentation du protocol Swagger ou équivalent
  \item Utilisation d'une API gateway gérant les problématique de sécurité , de limites d'utilisations selon configurations
  \item Architecture micro-service utilisant Docker
  \item Multi-API : architecture dynamique facilitant l'ajout de nouvelles API
  \item \textit{Continuous deployement}
\end{itemize}

L'aboutissement du projet consistait à remplacer les API utilisées actuellement par le portail par ces nouvelles API. De plus, le projet devrait s'intégrer aux autres développement de Fund KIS. Il devrait utiliser un ensemble de \textit{packages} communs : accès aux données, algorithmes, authentification. Les développement seraient faits en \textit{fullstack Javascript}: Mongo, Node, React, Angular...



% \vspace{5mm}
%
% \begin{figure}[!h]
% \centering
% \includegraphics[width=\linewidth]{documentation}
% \caption{Caption}
% \end{figure}
