Fund Kis dispose d'un grand nombre de données qu'elle collecte ou calcule. La majorité de ces informations sont rendues publiques à travers son portail \url{https://fundkis.com}. La mission réalisée au cours du stage a consisté à rendre ces informations accessibles à travers une \textit{API REST} implémentant les meilleures pratiques, dont :
\begin{itemize}[font=\color{blue}, label=\ding{43}]
  \item API auto-documentée avec portail de documentation ;
  \item implémentation du protocol Swagger ou équivalent ;
  \item utilisation d'une API gateway gérant les problématiques de sécurité et de limites d'utilisations selon configurations ;
  \item architecture micro-service utilisant Docker ;
  \item architecture multi-API, c'est-à-dire dynamique et facilitant l'ajout de nouvelles API ;
  \item \textit{Continuous deployement}.
\end{itemize}

L'aboutissement du projet consistait à remplacer les API utilisées auparavant par le portail par ces nouvelles API. De plus, le projet devait s'intégrer aux autres développements de Fund KIS. Il devait notamment utiliser un ensemble de \textit{packages} communs : accès aux données, algorithmes et authentification. Les développement devaient être faits en \textit{fullstack Javascript}: Mongo, Node, React, Angular...

Comme le travail à effectuer était plutôt conséquent pour être réalisé sereinement par un seul stagiaire de deuxième année en trois mois, Fund KIS a recruté deux stagiaires. J'ai donc travaillé avec une autre étudiante de l'EMSE sur le sujet.
