L'environnement de travail était très convivial pendant mon stage. Le premier jour du stage, notre tuteur nous a présenté brièvement les activités de l'entreprise pour situer le contexte du travail que nous avions à faire. Fund KIS était une petite entreprise de quatre personnes lorsque j'y ai fait mon stage. Nous étions deux stagiaires, ce qui qui faisait en tout six personnes. Il y avait deux espaces de travail : nous les deux stagiaires avions notre bureau avec un autre employé dans un espace et mon tuteur (Président de l'entreprise) et les deux autres employés avaient leur bureau dans l'autre espace. Je voyais donc tout le monde tous les jours et l'ambiace de travail était bien détendu.

\vspace{3mm}

Je pouvais poser facilement des questions à mon tuteur car il travaillait dans le bureau à côté et était très disponible pour mes interrogations. De plus je pouvais demander de l'aide Sylvain qui travaillais dans le même espace que nous les deux stagiaire. Au début c'était un peu difficile de se mettre à travailler et faire les tests directement dans la base de code déjà existante et j'avais tendance à tester ce que je faisais de façon complètement isolée. Mais après quelques entretiens avec mon tuteur, j'ai pu commencer à travailler en intégrant directement ce que je faisais dans les développements de l'entreprise.

\vspace{3mm}

Nous avons quelques fois, tous les six, déjeuné au restaurant à midi.
