{\it%
Une comptabilité saine consiste à classer chaque opération
dont on veut garder une trace. Ce programme permet de
sauvegarder, classer et traiter des opérations de comptabilité
traditionnelle, c'est-à-dire des flux d'argent entre les liquidités,
un ou plusieurs compte en banques, un ou plusieurs compte épargnes,
le tout classer et traiter suivant un schéma prévisionnel.

Plus simplement, l'ensemble des dépenses que l'on opère se
classent dans différentes catégories, par exemple un loyer pourra être considéré
comme une dépense dans une catégorie \og Dépenses fixes\fg ou 
\og Administratif\fg, et les dépenses courantes pour la nourriture dans
une catégorie \og Vie courante\fg ou \og Inévitable\fg,\dots

Chacune de ces catégories va donc représenter une certaine somme
dépensée tous les mois, ce qui permet d'allouer à cette catégorie
une somme au début du mois. Une comptabilité comparera l'état
des dépenses effectuées à un certain moment avec les provisions
prévues, et donc permet de garder un \oe il sur l'état des
finances.

Ce programme se propose de tracer l'ensemble des flux d'argent,
en traitant l'ensemble des opérations par rapport à des prévisions,
les suivant entre le liquides, un ou des comptes en banque et comptes
d'épargne. Ainsi on a une estimation précise de l'argent restant dans le/les
compte(s) et en poche.
}
